\documentclass[11pt, norsk, a4paper]{article}
\usepackage{charter,graphicx}

\usepackage{geometry}
\geometry{legalpaper, margin=1in}

\renewcommand{\baselinestretch}{1.2}
\usepackage{listings}
\usepackage{amsmath}
\usepackage{amsthm}
\usepackage{wrapfig}
\setlength{\parindent}{0in}
\usepackage{caption}
\usepackage{braket}
\usepackage{nccmath}
\usepackage{amssymb}
\usepackage{subfig}
\usepackage{listings}
\usepackage[makeroom]{cancel}
\usepackage{xcolor}

\newcommand\norm[1]{\lVert#1\rVert}
\newcommand\inner[1]{\langle#1\rangle}

\title{Sobolev inequalities}
\author{m.kristiansen}
\date{\today}

\begin{document}

\maketitle

\subsection*{Hölder's inequality}
Let (X, A, $\mu$) be a measure space and assume that $p,q \in (1, \infty)$ are conjugate i.e. $\frac{1}{p} + \frac{1}{q} = 1$. If $f, g$ are measurable functions, then 
\begin{align*}
    \int |fg| d\mu \leq \norm{f}_{L^p}\norm{g}_{L^q}
\end{align*}
This is a generalization of the Cauchy-Schwartz inequality ($p,q = 2$). \\
proof: see \cite{HIspaces}.

\subsection*{Poincare type inequalities}
\subsubsection*{Theorem 1}
Assume $\Omega$ is a bounded, open subset of $\mathbb{R}^n$ and suppose $u \in W_0^{1,p}$ for some $p\in [1, n)$. Then we have the estimate 
\begin{align*}
    \norm{u}_{L^q(\Omega)} \leq C(p, q, n, \Omega)\norm{Du}_{L^p(\Omega)}
\end{align*}
for each $q \in [1, p^*]$, where $p^*$ is the Sobolev conjugate of $p$, $p^* := \frac{np}{n-p} > p$. \\
proof: see \cite{PIevans}
 
\subsubsection*{Theorem 2}
Let $\Omega$ be a bounded, connected, open subset of $\mathbb{R^1}$ with a $C^1$ boundary $\partial \Omega$ and assume $p \in [1, \infty]$. Then, we have the estimate 
\begin{align*}
    \norm{u-(u)_\Omega}_{L^p} \leq C(n,p, \Omega) \norm{Du}_{L^p}
\end{align*}
for $u \in W^{1,p}$. $(u)_\Omega$ denotes the average of $u$ in $\Omega$. $(u)_\Omega := \frac{1}{|\Omega |} \int_\Omega u dx$. 
Note that we do not demand trace zero in this inequality. 
Proof: see \cite{PI2evans}




\begin{thebibliography}{9}
\bibitem{HIspaces}
Spaces - An introduction to Real Analysis, Tom L. Lindstrøm, p.278

\bibitem{PIevans}
Partial Differential Equations, Lawrence C. Evans, p. 279

\bibitem{PI2evans}
Partial Differential Equations, Lawrence C. Evans, p. 290

\end{thebibliography}
















\end{document}