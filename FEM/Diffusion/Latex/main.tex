\documentclass[11pt, norsk, a4paper]{article}
\usepackage{charter,graphicx}

\usepackage{geometry}
\geometry{legalpaper, margin=1in}

\renewcommand{\baselinestretch}{1.2}
\usepackage{listings}
\usepackage{amsmath}
\usepackage{amsthm}
\usepackage{wrapfig}
\setlength{\parindent}{0in}
\usepackage{caption}
\usepackage{braket}
\usepackage{nccmath}
\usepackage{amssymb}
\usepackage{subfig}
\usepackage{listings}
\usepackage[makeroom]{cancel}
\usepackage{xcolor}

\newcommand\norm[1]{\lVert#1\rVert}
\newcommand\inner[1]{\langle#1\rangle}

\title{Diffusion}
\author{m.kristiansen}
\date{\today}

\begin{document}

\maketitle


We consider the Possion problem 
\begin{align}
    -\Delta u &= f \hspace{1cm} \text{on} \hspace{0.1cm} \Omega \\
    \alpha u + \beta \frac{\partial u}{\partial n} &= g \hspace{1cm} \text{on} \hspace{0.1cm} \partial\Omega 
\end{align}
Where $n$ is the normal vector on $\partial \Omega$ and $\alpha, \beta \geq 0$ decides the specific boundary value problem. To numerically solve a given boundary value problem there are two ways continuing. By approximating $\Delta u$ with a finite difference scheme, or by rewriting (1) in the variational formulation. We will focus on the latter and start by multiplying (1) with a test function $v$ and inegrating over the domain $\Omega \subset \mathbb{R}^2$. Whenche, we obtain 
\begin{align*}
    -\int_{\Omega} \Delta uv dx = \int_\Omega fv dx 
\end{align*}
Applying Gauss-Green lemma to the LHS, we have 
\begin{align*}
    -\int_{\Omega} \Delta uv dx = \int_\Omega \Delta u \cdot \Delta v dx -\int_{\partial \Omega} v\frac{\partial u}{\partial n} dS
\end{align*}
Which gives us the form 
\begin{align*}
 \int_\Omega \Delta u \cdot \Delta v dx -\int_{\partial \Omega} v\frac{\partial u}{\partial n} dS = \int_\Omega fv dx
\end{align*}
Note that this gives us a direct way of implementing the Neumann type boundary conditions and we weaken our requirement on $u$ from $u \in C^2(\Omega)$ to $u \in C^1(\Omega)$. Lastly, we further reduce our requirement on $u$ by replacing $\Delta$ with the weak derivative $D$. \\
\\
\textit{Definition}\\
If $u, Du \in L^1_{loc}(\Omega)$ we say that $Du$ is the weak derivative of $u$ if 
\begin{align*}
    \int_\Omega uD\phi dx  = -\int_\Omega Du \phi dx 
\end{align*}
holds for all $\phi \in C^\infty_c(\Omega)$. \\
Weak derivatives are unique up to a set of measure zero. 
\begin{proof}
    Assume $Du, D\hat{u} \in L^1_{loc}(\Omega)$ satisfy 
    \begin{align*}
        \int_\Omega uD\phi dx = -\int_\Omega Du\phi dx = -\int_\Omega D\hat{u}\phi dx \hspace{1cm} \forall \phi \in C^\infty_c(\Omega). 
    \end{align*}
    Then 
    \begin{align*}
        \int_\Omega (D\hat{u} - Du) \phi dx = 0 
    \end{align*}
    Hence, $Du = D\hat{u}$ a.e.
\end{proof}
We have the (weak) variational formulation of (1)
\begin{align}
     \int_\Omega D u \cdot D v dx -\int_{\partial \Omega} v\frac{\partial u}{\partial n} dS = \int_\Omega fv dx
\end{align}
Consider the case where we multiply (1) with a function $v = u$, certainly we still want a unique solution $u$, hence we want to have a bound 
\begin{align*}
    \int_\Omega |Du|^2 dx < \infty 
\end{align*}
Which is exactly what is obtained searching for solutions $u \in H^1(\Omega)$ of (3) that holds for all $v \in H^1(\Omega)$. For the case of pure Dirichlet BC's ($\alpha = 1, \beta = 0$) we define solution and test spaces by 
\begin{align*}
    u \in \{u \in H^1(\Omega) : u = g \hspace{0.2cm} \text{on} \hspace{0.2cm}\partial \Omega\} = H^1_g(\Omega)\\
    v \in \{v \in H^1(\Omega) : v = 0 \hspace{0.2cm} \text{on} \hspace{0.2cm} \partial \Omega\} = H^1_0(\Omega)
\end{align*}
Notice that, since $v = 0$ on $\partial \Omega$ the Neumann term vanishes in (3). 
\subsection*{Lax-Milgram}
If we can show boundedness and ellipticity of the bilinear operator 
\begin{align*}
    B[u,v] = \int_\Omega Du\cdot Dv dx
\end{align*}
and demanding $f \in L^2(\Omega)$ we have, by the Lax-Milgram theorem, that there exists a unique solution $u$ to the problem
$$B[u,v] = \inner{f,v}_{L^2}.$$
To show boundedness we use the Cauchy-Schwartz inequality 
\begin{align*}
    |B[u,v]| = |\inner{Du, Dv}_{L^2}| \leq \norm{Du}_{L^2}\norm{Du}_{L^2}
\end{align*}
and since $\norm{u}_{H^1} = \norm{u}_{L^2}+\norm{Du}_{L^2}$ we clearly have 
\begin{align*}
    |B[u,v]|\leq \norm{u}_{H^1}\norm{v}_{H^1}.
\end{align*}

\subsection*{Homogeneous Dirchlet}
To show ellipticity we want to use Poincare's inequality (Evens, page 279). Considering homogeneous Dirichlet we have $u,v \in H^1_0(\Omega) (Tu = 0$ on $\partial \Omega)$ and assume further that $\Omega$ to be a bounded open subset of $R^n$. Then, by Poincare 
\begin{align*}
    \norm{u}_{L^2(\Omega} \leq C\norm{Du}_{L^2(\Omega)} 
\end{align*}
whence, 
\begin{align*}
    \norm{u}_{H^1} \leq C\norm{Du}_{L^2}+\norm{Du}_{L^2} = (C+1)\norm{Du}_{L^2}
\end{align*}
Which gives us 
\begin{align*}
    |B[u,u]| \geq \frac{1}{(1+C)^2} \norm{u}_{H^1}
\end{align*}
Consider Friedrich inequality to further estimate $C$.

\subsection*{non-homogeneous Dirichlet}
For nonzero Dirichlet boundary ($g \neq 0)$ conditions we no longer have $u \in H_0^1(\Omega)$ and can no longer apply Poincare's inequality. To obtain ellipticity we define $u = u_0 + u_g$, where $u_0 \in H^1_0(\Omega)$ i.e. trace zero on the boundary and $u_g$ that takes the values of $g$ on the boundary. Our variational problem then becomes; find $u_0 \in H^1_0(\Omega) $ s.t. 
\begin{align}
    \int_{\Omega} Du_0\cdot Dv dx = \int_{\Omega} fv -Du_g\cdot Dv dx 
\end{align}
holds for all $v\in H^1_0(\Omega)$. The solution to our original boundary problem is then given by $u = u^h_0+u_g$ where $u^h_0$ solves (4). Requiring $u_g \in H^1$ the RHS of (4) represents a bounded linear functional from $H \to \mathbb{R}$ and we know that the LHS represents a bounded and elliptic bilinear form. Hence by Lax - Milgram we have well-posedness. There are different ways of choosing $u_g = g$ on $\partial \Omega$ $\in H^1(\Omega)$. \cite{Master}



\subsection*{Neumann}
For pure Neumann BC's ($\alpha = 0, \beta = 1$) (3) becomes 
\begin{align*}
     \int_\Omega D u \cdot D v dx -\int_{\partial \Omega} vg dS = \int_\Omega fv dx
\end{align*}
Testing, this with $v = 1$ leads to the compatibility condition 
\begin{align*}
    -\int_{\partial \Omega} g dS = \int_\Omega f dx
\end{align*}
Using Gauss-Greens lemma we can rewrite this as 
\begin{align*}
    -\int_\Omega Du dx = \int_\Omega f dx.
\end{align*}
From this we see that we see that the addition of any constant to $u$ also solves our boundary problem. For us to have a unique solution we fix the constant by demanding 
\begin{align*}
    \int_\Omega v dx = \int_\Omega u dx = 0.
\end{align*}
Our test and trial space thus becomes 
\begin{align*}
    u,v \in \{u \in H^1(\Omega) : \int_\Omega u dx = 0\} = H^1_n(\Omega)
\end{align*}
We have already seen that $a(u,v) = \int Du \cdot Dv dx$ is bounded for $u,v \in H^1_n(\Omega) \subset H^1(\Omega)$, and by moving the boundary term to the RHS we have 
\begin{align*}
    L(v) = \int_\Omega fv -\int_{\partial \Omega}gv
\end{align*}
Which is a bounded linear functional when $f \in L^2(\Omega)$ and $g \in L^2(\partial \Omega)$. It remain to show that $a(u,v)$ is coercive. From theorem 2 in Sobolev inequalities, we have 
\begin{align*}
    \norm{u-(u)_\Omega}_{L^2} \leq C\norm{Du}_{L^2}
\end{align*}
for $u \in H^1(\Omega)$ and since $(u)_\Omega =\int_\Omega u dx = 0 $ we obtain the same inequality as for the homogeneous Dirichlet case. Note, however that here we do not demand trace zero on the boundary. Coercivity/elliptisity follows from the previous derivation.  
\subsection*{Robin boundary conditions}
Not sure if this is needed. 


\begin{thebibliography}{9}




\end{thebibliography}

\bibitem{Master}
{https://www.math.uzh.ch/compmath/fileadmin/user/stas/compmath/Abschlussarbeiten/zueger-francesco-masterarbeit.pdf}













\end{document}
